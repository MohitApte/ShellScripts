\documentclass[11pt,a4paper]{report}
\usepackage[utf8]{inputenc}
\usepackage{amsmath}
\usepackage{amsfonts}
\usepackage{amssymb}
\usepackage{graphicx}
\usepackage{graphicx}
\usepackage{csvsimple}
\graphicspath{ {./images/} }
\usepackage[left=2cm,right=2cm,top=2cm,bottom=2cm]{geometry}
\author{MOHIT APTE MIS-112103012}
\title{DTL ASSIGNMENT 4}
\date{2022-08-12}
\begin{document}
\maketitle

\begin{LARGE}

\begin{center}
\textbf Serum homocysteine levels in patients with retinal vein occlusion
\end{center}
\end{LARGE}


Purpose: To find out role of high serum homocysteine levels in retinal vein occlusion patients at Dr. D.Y. Patil medical College.
Design: A matched case control type of study was conducted from 2016 to 2018.
Materials and Methods: Total serum homocysteine (tHcy) was measured in patients coming at Dr. D.Y. Patil medical college, aged 20
years and above. We evaluated the presence of high homocysteine levels in patients with retinal vein occlusion. We evaluated serum
homocysteine levels in 50 patients with retinal vein occlusion coming to our clinic. Control subjects consisted of age and sex matched
patients that were referred to our clinic for retinal disease other than vascular occlusion.~\cite{a} homocysteine levels between 4 µmol/L to 15
µmol/L were considered normal. High homocysteine level was defined as a total serum homocysteine level above 15 µmol/L.
Results: The mean serum homocysteine level were 13.80+/-8.08 µmol/L (range, 4–33 µmol/L) for cases, and 6.43+/-1.38 µmol/L (range,
4–10 µmol/L) for controls.~\cite{b} The mean serum homocysteine levels in cases were more than double that of controls. This difference was very
highly statistically significant. Out of the 50 patients with retinal vein occlusion, 13 (26.00%) patients had serum homocysteine levels
above normal levels as compared to the controls where none out of the 50 patients had homocysteine levels above normal. (Chi square =
14.94, d.f.=1, p<0.001), This difference was very highly statistically significant. 38 (60%) patients were present with hypertension in both
cases and control groups.~\cite{c}
Conclusion: High homocysteine levels is a statistically significant risk factor for retinal vein occlusion and it should be evaluated in every
patient with retinal vein occlusion.~\cite{d}


\begin{thebibliography} {}

\bibitem {a}Mudd SH, Levy HL, Skovby F. Disorders of transsulfuration.
In: Scriver CR, Beaudet AL, Sly WS, et al, editors. The
metabolic and molecular bases of inherited disease. New York:
McGraw-Hill, 1995:1279-1327.

\bibitem{b} Frosst P, Blom HJ, Milos R. A candidate genetic risk factor for
vascular disease: a common mutation in
methylenetetrahydrofolate reductase. Nat Genet 1995;10:111–
113

\bibitem{c} Nygård O, Refsum H, Ueland PM. Major lifestyle
determinants of plasma homocysteine distribution: The
Hordaland Homocysteine Study. Am J Clin Nutr 1998;67:263–
270

\bibitem{d} Morgan SL, Baggott JE, Lee JY. Folic acid supplementation
prevents deficient blood folate levels and
hyperhomocysteinemia during long-term, low dose
methotrexate therapy for rheumatoid arthritis: implications for
cardiovascular disease prevention. J Rheumatol 1998;25:441–
446




\end{thebibliography} 
\newpage
\includegraphics[width=\textwidth]{stock}
\includegraphics[width=\textwidth]{stock3}
\end{document}