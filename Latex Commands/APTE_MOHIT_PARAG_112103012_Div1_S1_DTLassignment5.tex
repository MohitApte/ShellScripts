\documentclass{article}
\usepackage{multicol}
\usepackage{graphicx}
\usepackage{csvsimple}
\graphicspath{ {./images/} }
\title{Research Paper}
\author{Mohit Apte MIS - 112103012}
\begin{document}
\maketitle
\newpage
\begin{center}
\begin{large}
\textbf Serum homocysteine levels in patients with retinal vein occlusion

\end{large}
\end{center}
\hrulefill
\newline
\textbf Abstract:
\newline
\textbf Purpose: 
To find out role of high serum homocysteine levels in retinal vein occlusion patients at Dr. D.Y. Patil medical College.

\textbf Design: 
A matched case control type of study was conducted from 2016 to 2018.

\textbf Materials and Methods: 
Total serum homocysteine (tHcy) was measured in patients coming at Dr. D.Y. Patil medical college, aged 20
years and above. We evaluated the presence of high homocysteine levels in patients with retinal vein occlusion. We evaluated serum
homocysteine levels in 50 patients with retinal vein occlusion coming to our clinic. Control subjects consisted of age and sex matched
patients that were referred to our clinic for retinal disease other than vascular occlusion. homocysteine levels between 4 µmol/L to 15
µmol/L were considered normal. High homocysteine level was defined as a total serum homocysteine level above 15 µmol/L.

\textbf Results: 
The mean serum homocysteine level were 13.80+/-8.08 µmol/L (range, 4–33 µmol/L) for cases, and 6.43+/-1.38 µmol/L (range,
4–10 µmol/L) for controls. The mean serum homocysteine levels in cases were more than double that of controls. This difference was very
highly statistically significant. Out of the 50 patients with retinal vein occlusion, 13 (26.00%) patients had serum homocysteine levels
above normal levels as compared to the controls where none out of the 50 patients had homocysteine levels above normal. (Chi square =
14.94, d.f.=1, p<0.001), This difference was very highly statistically significant. 38 (60%) patients were present with hypertension in both
cases and control groups.
\textbf Conclusion: 
High homocysteine levels is a statistically significant risk factor for retinal vein occlusion and it should be evaluated in every
patient with retinal vein occlusion

\hrulefill
\newline
\begin{multicols*}{2}
[
\textbf Introduction:
]
Hyperhomocystienemia is considered as increase of the
homocysteine level in blood, serum or plasma.
Homocysteine levels in serum between 4 to 15
micromoles/liter (µmol/L) is considered normal. Levels
above 15 µmol/L is considered high. Homocysteine levels
on an average are below 10 µmol/L.~\cite{b} Hyperhomocysteinemia
can be moderate type, intermediate type, and severe type
depending on the level of serum homocysteine: Moderate
(15 to 30 µmol/L), Intermediate (30 to 100 µmol/L), Severe
(greater than 100 µmol/L).1
Causes of high serum homocysteine levels can be
genetic factors determinants, lifestyle (vegetarian diet, high
coffee consumption), medicines (methotrexate), etc.2-5
Inadequate levels folate or of the B vitamins account for
very high number of cases of hyperhomocysteinemia.~\cite{b}
6
McCully and Wilson proposed the theory of high serum
homocysteine levels leading to arteriosclerosis.
7 Research
relating high homocysteine levels with vascular disorders
has been done extensively in the last 10 to 15 years.~\cite{a}
8,9
Increase in serum homocysteine level has been associated
with myocardial infarction, stroke and carotid wall
thickening in adults with no history of atherosclerotic
disease. Various European studies has shown that high
homocysteine levels are an independent risk factor for
vascular disease.~\cite{a}
9 High serum homocysteine level is a high
risk factor for venous thrombosis, has been shown by many
studies.~\cite{d}
10,11 A large multicentre study (The eye disease case
control study group), have suggested that cardiovascular
risk factors are also the risk factors for retinal vein
occlusions.
12 But they did not include measurement of
serum homocysteine levels in their study. Our study seeks to determine whether high serum homocysteine levels is a risk
factor for retinal vein occlusion. Limited studies on this
issue are available in the India. Our study estimated the role
of high serum homocysteine levels in the occurrence of
retinal vein occlusion.~\cite{c}
\newline
\newline
\textbf Materials and Methods:
\newline
Our study was conducted between 2016 and 2018. We
took a verbal consent of the patients. The study type was a
matched pair case control study. The difference in the mean
levels of serum homocysteine among cases group and
controls group was around 14 umol/L and this was used to
calculate the sample size. ~\cite{d}The standard deviation in cases
group as high as 26 umol/L and in control group as high as
10 umol/L was considered. We selected 50 patients with
Retinal Vein Occlusion and 50 cases without Retinal vein
occlusion. This gave an 80% precision to our study. We
considered 1:1 ratio of age and sex matched cases and
controls.~\cite{c} Patients aged 20 years and older, diagnosed at our
clinic between 2016 and 2018, and having Retinal vein
occlusion in were included in cases group. Age- and sex
matched patients with retinal disease without Retinal Vein
Occlusion were included in the controls group. Diabetics,
hyperthyroidism, patients having tobacco or tobacco related
products, patients consuming alcohol, patients having
undergone eye surgery in the last 1 year, pregnant women,
were excluded from the study.~\cite{c} Senior ophthalmologist and
ophthalmology residents undergoing training in
ophthalmology were our field staff. Visual acuity for
distance, and best corrected visual acuity was noted using
Snellen’s literate chart. ~\cite{d}
\newline
\newline
\textbf Results:
We tested serum homocysteine levels in 50 patients
with retinal vein occlusion and 50 patients with retinal
diseases without retinal vein occlusion. Table 1 shows the
characteristics of cases and control group patients. The
mean serum homocysteine level was 13.80+/-8.08 µmol/L
(range, 4–33 µmol/L) in the cases group, and 6.43+/-1.38
µmol/L (range, 4–10 µmol/L) for control group. The mean
serum homocysteine levels in cases group were more than
double that of controls. This difference was very highly
statistically significant. Table 2 shows the serum
homocysteine levels of patients in cases and control group.
Serum homocysteine levels above 15 µmol/L was
considered as hyperhomocysteinemia. 13 (26.00%) out of
the 50 patients with retinal vein occlusion had
hyperhomocysteinemia as compared to the control group
where none of the 50 patients had homocysteine levels
above normal. ~\cite{d}(Chi square = 14.94, d.f.=1, p<0.001). This
difference was very highly statistically significant.
\end{multicols*}


\newpage

\begin{thebibliography} {}

\bibitem {a}Mudd SH, Levy HL, Skovby F. Disorders of transsulfuration.
In: Scriver CR, Beaudet AL, Sly WS, et al, editors. The
metabolic and molecular bases of inherited disease. New York:
McGraw-Hill, 1995:1279-1327.

\bibitem{b} Frosst P, Blom HJ, Milos R. A candidate genetic risk factor for
vascular disease: a common mutation in
methylenetetrahydrofolate reductase. Nat Genet 1995;10:111–
113

\bibitem{c} Nygård O, Refsum H, Ueland PM. Major lifestyle
determinants of plasma homocysteine distribution: The
Hordaland Homocysteine Study. Am J Clin Nutr 1998;67:263–
270

\bibitem{d} Morgan SL, Baggott JE, Lee JY. Folic acid supplementation
prevents deficient blood folate levels and
hyperhomocysteinemia during long-term, low dose
methotrexate therapy for rheumatoid arthritis: implications for
cardiovascular disease prevention. J Rheumatol 1998;25:441–
446




\end{thebibliography} 
\includegraphics[width=\textwidth]{stock}
\includegraphics[width=\textwidth]{stock3}

\end{document}
